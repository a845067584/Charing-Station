Hence, we studied the optimization problem of how to deploy charging stations. Existing works mainly fall into the domain of bike-sharing. \cite{Bao:2017} provieds a data-driven apporoach to deal with bike lane construction problem. It takes government constraints of planning bike lanes, such as budget limitations, construction convenience and bike lane utilization into consideration to formulate the problem. Furthermore, the problem is proved to be NP-hard so that they propose a greedy network expansion algorithm to help work out a scalable and approximate solution to bike lane planning problem. The approach performs well in the given problem, however it doesn't make use of learning models. \cite{Li:2018:DBR} introduces a reinforcement learning algorithm to help solve the problem of repositioning sharing-bikes. First it uses an inner-balance clustering algorithem to cluster stations into groups, then the reinforcement learning algorithm is conducted in each group to learn a reposition policy. They make a good use of spatio-temporal data while don't take advantages of useful geographical and station-self features. 

Current works of location selection are usually based on the flow prediction of a single station. Futhermore, they rely heavily on the historical data. \cite{Yang:2016:MMP} introduces a model for bicycle mobility prediction. It relis on historical bike-sharing data and a per-station basis with sub-hour granularity. It makes use of the randonm forest prediction model to implement their experiments and obtain a rather good result. \cite{Liu:2016:CTP} gives an optimization to this problem. In this work, traffic prediction no longer focus on the history data only, but can use location-based socail media to collect a much larger area of the traffic data for predicting traffic conditions. \cite{Shen:2018:SNS} is also a good example of prediction model for spatio-temporal mobility event. It encodes each POI's spatio and temporal dependencies rather than neglect the correlations between POIs. In this paper, we argue that the surrounding point of interests(POIs), distances to important POIs(e.g, metro stations, estates, etc.), station charging price, AC/DC station types as well as whether a station is private of public for use, play important roles in selecting the optimal location for stations.