With the development of electric vehicles, there comes a great need of charging stations for the recharging demand. However, where to locate a station and what are the main elements that operators should take into account when planning a setting, are still bothering problems that wait to be solved. In a common view, a better place to set a station ought to gurantee a relatively higher use rate of that station. Therefore, the problem changes into how to gain a higher use rate, and what are the factors that have important impact on it. In this paper, we propose a time frame based prediction framework of use rate for charging stations in Shanghai. The approach proposed in this paper takes both station's geographical information, such as longitude, latitude and Point of Interests(POIs), and working elements including price, AC/DC type and private or public to use, into consideration. We also seperate our datasets into different time frames including total time, weekday, weekend, morning, evening, moring\_rush hours, evening\_rush hours and travel hours. The aim is to classify whether it's a high-use-rate station or a low-use-rate one during different time periods. Experimental results show that our method performs well on LR, Random Forest, SVM and XGBOOST, which demonstrates that featrues as geographical information and working elements play an important role in use rate of charging stations.
