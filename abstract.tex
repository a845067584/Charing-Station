With the development of electric vehicles, there comes a great need of charging stations for the recharging demand. However, where to locate a station and what are the main elements that operators should take into account when planning a setting, are still bothering problems that wait to be solved. In a common view, a better place to set a station ought to gurantee a relatively higher use rate of that station. Therefore, the problem changes into how to gain a higher use rate, and what are the factors that have important impact on it. In this paper, we propose a spatio temporal data based prediction framework of use rate for charging stations in Shanghai. The approach proposed in this paper takes both station's geographical information, such as longitude, latitude, surrounding Point of Interests(POIs), and working elements including price, AC/DC type and private or public to use, into consideration. After prerocessing, we seperate our datasets into urban area, suburb area and different time frames including total time, weekday, weekend, morning, evening, moring\_rush hours, evening\_rush hours and travel hours. We evaluate our method in two tasks, including districts prediction and time frames prediction. The aim is to classify what the level of a charging station's use rate is in different area districts and during different time periods. Experimental results show that our method performs well on SVM, Random Forest and MLP(ANN), which demonstrates that featrues as geographical information and working elements play important roles in use rate of charging stations. What's more, in the second task, we also find that the prediction accuracy is strongly attached to time span that a time frame covers.
